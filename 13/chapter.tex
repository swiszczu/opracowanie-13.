\setcounter{chapter}{12}
\chapter{Programowanie miktrokontrolerów i mikroprocesorów}
\PartialToc
%\startcontents[chapters]
%\printcontents[chapters]{}{1}{\section*{\contentsname}}
\setcounter{section}{215}

%-----------------
\answer
{Ile rejestrów 8-bitowych dostępnych dla programisty znajduje się w procesorach z rodziny x86?}
{6}
{F}
{8}
{
	Innej poprawnej nie ma. Rysunek poniżej
	\begin{center}
		\includegraphics[width=6cm]{13/img/x86-registers.png}
		\captionof{figure}{Struktura rejestrów procka x86}
	\end{center}
}
%-----------------
\answer
{Jaki tryb adresowania wykorzystuje rozkaz ADDL (\%ebx),\%eax?\\}
{bezposredni}
{F}
{pośredni}
{
Główny podział:
\begin{itemize}
	\item bezpośredni (ang. direct)
	\item pośredni (ang. indirect)
\end{itemize}
Bardziej szczegółowo:
\begin{itemize}
	\item rejestrowy (ang. register)
	\item rejestrowy pośredni (ang. register indirect)
	\item z przesunieciem (indeksowanie) (ang. displacement (indexed))
	\item stosowy (ang. stack)
\end{itemize}
}
\begin{lstlisting}
bezposredni
movl 0x01, %eax
posredni
movl (%ebx), %eax
movl (%esi,%edi,1), %eax

\end{lstlisting}
%-----------------
\answer
{Jaka instrukcja jest równowazna w działaniu do instrukcji SHL \$1,\%eax?}
{SAL \$1,\%eax}
{T}
{}
{\\\textbf{SAL} - przesunięcie arytmetyczne w lewo\\
\textbf{SHL} - przesunięcie bitowe w lewo.\\
Przesuwanie w lewo czyli w stronę flagi CF. Przy ostatnim przesunięciu flaga zostanie wyzerowana więc nie ma znaczenia dla wyniku czy stosujemy arytmetyczne czy bitowe przesuniecie.}
\begin{lstlisting}
MOVW $0xFF00, %ax   #=> 0xff00 65280   # 1 1111 1111 0000 0000
SHL   $1, %eax      #=> 0x1fe00 130560 # 1 1111 1110 0000 0000
##
MOVW $0xFF00, %ax   #=> 0xff00 65280   # 1 1111 1111 0000 0000
SAL   $1, %eax      #=> 0x1fe00 130560 # 1 1111 1110 0000 0000

\end{lstlisting}
%-----------------%-----------------
\answer
{Która z ponizszych instrukcji dotyczy operacji na blokach danych?}
{STC}
{F}
{np. \textbf{MOV, PUSH, POP, XCHG, XLAT}}
{\begin{itemize}
\item[] \textbf{STC} (STC/CLC - set carry / clear carry. Do flagi CF wstaw 1 lub 0, odpowiednio). \\FLAGI: \textbf{CF, OF, SF, ZF, IF, PF, DF} służą przede wszystkim do sterownia znacznikami np. do badania wyniku ostatniego przekształcenia (przepełnienie / wynik 0).
\end{itemize}}
%More - wykłądy Bublińskiego, Język asembler dla każdego Bogdan Drozdowski
%-----------------%-----------------
\answer
{Według jakiej reguły moze być dokonywana konwersja do liczby całkowitej w jednostce FPU (Floating Point Unit)?
}
{round down}
{T}
{\\
	00 - round to nearest\\
	01 - round down\\
	10 - round up\\
	11 - round toward zero\\}
{Powyżej możliwe ustawienia kontroli zaokrągleń w FPU. \\Źródło wykłady FTP wykłd 9. s16}
%-----------------%-----------------
\answer
{Ile razy wykona sie pętla zbudowana w oparciu o instrukcję LOOP, jesli przed jej rozpoczeciem zawartosc rejestru \%ecx była równa 0? }
{$2^{32}$}
{T}
{Chyba poprawne?}
{rozkaz loop to petla for(\%ecx - -;\%ecx!=0;)\{\%ip+=disp\}. \\Źródło wykłady FTP}
%-----------------%-----------------
\answer
{Ile razy zawartosc rejestru \%ah zostanie zapisana do pamieci poprzez uzycie instrukcji REP STOSB, jezeli przed jej wykonaniem zawartosc rejestru \%ecx była równa x?}
{1}
{F}
{hmm czyżby chodzilo o zawartość al?, jeśli nie to zero razy bo STOSB wpisze AL nie AH, a jesli mial na mysli AL to x razy}
{REP powtarza instrukcje x razy, w tym wypadku wezmie pod uwage rejestr CL jako licznik}
%-----------------%-----------------
\answer
{
	Jaka bedzie zawartosc rejestru \%eax po sekwencji rozkazów?\\
	MOVL \$0xFFFF0000 ,\%eax\\
	NEG \%eax
}
{0x00000000}
{F}
{0x00010000}
{
	\lstinputlisting{13/ideone_5Uwlpn.s}
}
%-----------------%-----------------
\answer
{Jaka bedzie zawartosc rejestru \%al po sekwencji rozkazów? \\
	MOVW \$0xFF00 ,\%ax\\
	ADCB \%ah ,\%al\\
	ADCB \%ah ,\%al\\
	}
{nieokreslona}
{F}
{0xFFFE}
{
	\lstinputlisting{13/ideone_13223.s}
}
%-----------------%-----------------
\answer
{Na jakim rodzaju schematu pokazane sa połaczenia elektryczne w układzie opartym na mikrokontrolerze?}
{ideowym}
{T}
{?}
{}
%-----------------%-----------------
\answer
{W jakim rodzaju pamieci mikrokontrolera uzytkownik zwykle zapisuje kod programu?}
{DRAM}
{F}
{
	\begin{itemize}
		\item EEPROM, Flash - zwykle
		\item EPROM, MRAM lub FRAM
	\end{itemize}
}
\\{Powyżej wymienione pamięci są pamięciami nieulotnymi, i tam umieszczamy kod programu.\\DRAM i SRAM są pamięciami ulotnymi, czyli po wyłączeniu zasilania kod przepada.}
%-----------------%-----------------
\answer
{Jakie elementy wystepujace w mikrokontrolerach nie wystepuja w mikroprocesorach?}
{ADC}
{T}
{}
{\begin{itemize}
\item Mikroprocesor to tylko:\\- \textbf{ALU} (Arithmetic Logic Unit) - jednostka licząca\\ - \textbf{CU} (Control Unit) - układ sterujący pracą\\ - \textbf{Rejestry}: PC, IR, SP - potrzebne przy wykonywaniu operacji obliczeniowych\\
\item Mikrokontroler to układ w pełni autonomiczny, potrzebujący tylko zasilanie do pracy. Posiada w sobie procesor oraz peryferia np.:\\- pamięć danych do przechowywania kodu programu i zmiennych,\\- porty wejścia-wyjscia np: \textbf{ADC} (Analog-to-digital converter), \textbf{DAC},\\- układy czasowo-licznikowe, \\- kontrolery przerwań,\\- kontrolery transmisji\textbf{ UART, SPI, I2C, USB, CAN, 1-Wire} etc.,\\- zegar czasu rzeczywistego \textbf{RTC},\\- watchdog.
\end{itemize}}
%-----------------%-----------------
\answer
{Czy jezyk maszynowy jest tozsamy z jezykiem asemblera? }
{nie}
{T}
{}
{Język ASM to tekstowe odpowiedniki wewnętrznych rozkazów procesora (tzw. mnemoniki), kod maszynowy to po prostu liczby}
%-----------------%-----------------
\answer
{Jakie narzedzie słuzy do zamiany kodu napisanego w jezyku asemblera na kod maszynowy? }
{assembler}
{T}
{}
{}
%-----------------%-----------------
\answer
{Które z narzedzi nie umozliwia stworzenia kodu na mikrokontroler z rodziny AVR?}
{Microsoft Visual Studio
}
{F}
{Wszystkie narzędzia które nie wspierają Make, czyli da się np. w Eclipse, dedykowane do AVR jest AVR Studio}
{http://www.instructables.com/id/Use-Visual-Studio-2010-to-Compile-AVR-Hex-Files/}
%-----------------%-----------------